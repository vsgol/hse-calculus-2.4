\Subsection{Лекция 2}


\begin{example}\thmslashn
	
	Задача: есть фотография $4000 \times 6000 = 24 \cdot 10^6$, на каждый пиксель на по $24$ бита, а это $24^2 \cdot 10^6 \approx 5 \cdot 10^8 \approx 60$ Мегабайт, но не каждый набор пикселей это фотография, обычно соседние пиксели похожих цветов. Хотим приближать набор векторов меньшей размерности приблизить к фотографии.
	
\end{example}

\begin{lemma}\thmslashn
	
	Если $\sum\limits_{n = 1}^\infty x_n$ сходится, то $\left\langle \sum\limits_{n = 1}^\infty x_n, y \right\rangle = \sum\limits_{n = 1}^\infty \left\langle x_n, y \right\rangle$
	
\end{lemma}

\begin{proof}\thmslashn
	
	$S_n = \sum\limits_{k = 1}^n x_k \to S \Rightarrow \left\langle S_n, y \right\rangle \to \left\langle S, y \right\rangle$

	Но также $\left\langle S_n, y \right\rangle = \left\langle \sum\limits_{k = 1}^n x_k, y \right\rangle = \sum\limits_{k = 1}^n \left\langle x_n, y \right\rangle $
	
\end{proof}

\begin{definition}\thmslashn
	
	$x, y$ -- ортогональны $x\perp y$, если $\left\langle x, y \right\rangle = 0$	
	
\end{definition}

\begin{definition}\thmslashn
	
	$\sum\limits_{n = 1}^\infty x_n$ -- ортогональный ряд, если $x_k\perp x_n$ при $k \not = n$

\end{definition}

\begin{theorem}\thmslashn 
	
	В гильбертовом  пространстве ортогональный ряд $\sum\limits_{n = 1}^\infty x_n$ сходится $\Leftrightarrow \sum\limits_{n = 1}^\infty \norm{x_n}^2 < + \infty$
	
	В этом случае $\norm{\sum\limits_{n = 1}^\infty x_n}^2 = \sum\limits_{n = 1}^\infty \norm{x_n}^2$
	
\end{theorem}

\begin{proof}\thmslashn
	
	$S_n = \sum\limits_{k = 1}^n x_k   \quad C_n =  \sum\limits_{k = 1}^n \norm{x_k}^2$
	
	Ряд $ \sum\limits_{n = 1}^\infty x_n \Leftrightarrow S_n$ -- фундаментальная $\Leftrightarrow$
	
	$\forall \varepsilon > 0 \exists N\,\, \forall m \geqslant n \geqslant N \quad \norm{S_m - S_n}^2 < \varepsilon$
	
	$\norm{S_m - S_n}^2 = \left\langle \sum\limits_{k = n+1}^m x_k,  \sum\limits_{j = n+1}^m x_j\right\rangle =  \sum\limits_{k = n+1}^m \sum\limits_{j = n+1}^m \left\langle x_k, x_j \right\rangle =  \sum\limits_{k = n+1}^m \norm{x_k}^2 = C_m - C_n \Leftrightarrow C_n$ -- фундаментальная $\Leftrightarrow \sum\limits_{n = 1}^\infty \norm{x_n}^2 < \infty$
	
	$ \left\langle S_n, S_n \right\rangle = \norm{S_n}^2 = C_n \to \sum\limits_{n = 1}^\infty \norm{x_n}^2$
	
	$\left\langle S_n, S_n \right\rangle  \to \left\langle S, S \right\rangle $
	
\end{proof}

\begin{consequence}\thmslashn
	
	Если ортогональный ряд $\sum\limits_{n = 1}^\infty x_n$ сходится, $\varphi: \N \to \N$ перестановка, то $\sum\limits_{n = 1}^\infty x_{\phi{n}}$ сходится к той же сумме
	
\end{consequence}

\begin{proof}\thmslashn
	
	$S = \sum x_т$ -- сходится $\Leftrightarrow \sum \norm{x_{\varphi(n)}}^2  \sum \norm{x_n}^2 < \infty \Leftrightarrow \tilde{S} = \sum x_{\varphi(n)}$ -- сходится
	
	$\norm{S - \tilde{S}}^2 = \left\langle S - \tilde{S}, S - \tilde{S} \right\rangle = \left\langle S - \tilde{S}, \sum x_{n} - \sum x_{\varphi(n)} \right\rangle = \sum \left\langle S - \tilde{S}, x_n - x_{\varphi(n)} \right\rangle = \sum\limits_{n = 1}^\infty \sum\limits_{k = 1}^\infty \left\langle x_k - x_{\varphi(k)}, x_n - x_{\varphi(n)} \right\rangle = \sum\limits_{n = 1}^\infty \left(  \left\langle x_n, x_n \right\rangle - \left\langle x_n, x_n \right\rangle - \left\langle x_{\varphi(n)}, x_{\varphi(n)} \right\rangle +  \left\langle x_{\varphi(n)}, x_{\varphi(n)} \right\rangle \right) = 0$
	
\end{proof}

\begin{definition}\thmslashn
	
	$x_ 1, x_2, \ldots$ -- векторы в $H$
	
	-- ортогональная система, если $x_k \perp x_n$ при $k\not = n$ и $x_n \not = 0\,\,\forall n$
	-- ортонормированная система, если $\left\langle x_k, x_n \right\rangle  = \begin{cases}
	0 & \text{ при } k\not = n\\
	1 & \text{ при } k = n\\
	\end{cases}$

\end{definition}

\begin{remark}\thmslashn
	
	Ортогональная система линейно независима
	
\end{remark}

\begin{proof}\thmslashn
	
	От противного $c_1x_1 + c_2 x_2 + \ldots + c_nx_n = 0 \Rightarrow 0 = \left\langle c_1x_1 + c_2 x_2 + \ldots + c_nx_n, x_k \right\rangle = \sum\limits_{j = 1}^nc_k  \left\langle x_j, x_k \right\rangle = c_k\left\langle x_k, x_k \right\rangle = c_k\norm{x_k}^2 \Rightarrow c_k = 0$
	
\end{proof}

\begin{example}[ортогональных систем]\thmslashn
	
	\begin{enumerate}
		\item 
		$l^2\quad e_n = (0, \ldots, 0, 1, 0, \ldots)$ -- ортонормированная система
		
		\item
		$L^2[0, 2\pi]\quad 1, \cos t, \sin t, \cos 2t, \sin 2t, \ldots$ -- ортогональная система
		
		
		\item
		$L^2[0, 2\pi]\quad e^{int}, n\in \Z$ -- ортогональная система
		
		$\left\langle e^{int}, e^{ikt} \right\rangle = \int\limits_{0}^{2\pi} e^{int}\cdot e^{-ikt}\,dt = \begin{cases}
		2\pi & \text{ если } n = k\\
		\left. \frac{e^{i(n-k)t}}{n-k}\right|_{0}^{2\pi} = 0 & \text{ если } n \not= k\\
		\end{cases}$
		
		\item
		$L^2[0, \pi] \quad 1, \cos t, \cos 2t, \ldots$ ортогональная система
		$\quad\quad \sin t, \sin 2t, \ldots$ ортогональная система
	\end{enumerate}
	
\end{example}

\begin{theorem}\thmslashn 
	
	Пусть $e_1, e_2, \ldots$ -- ортогональная система
	
	$x = \sum\limits_{n = 1}^{\infty} c_ne_n$ сходящийся ряд
	
	Тогда $c_n = \dfrac{\left\langle x, e_n \right\rangle }{\norm{e_n}^2}$
	
\end{theorem}

\begin{proof}\thmslashn
	
	$\left\langle x, e_n \right\rangle = \left\langle \sum\limits_{k = 1}^\infty c_ke_k, e_n \right\rangle =  \sum\limits_{k = 1}^\infty  c_k \left\langle e_k, e_n \right\rangle = c_n\left\langle e_n, e_n \right\rangle = c_n\norm{e_n}^2$
	
\end{proof}

\begin{definition}\thmslashn
	
	Пусть $e_1, e_2, \ldots$ ортогональная система, $x \in H$
	
	$c_n(x) = \dfrac{\left\langle x, e_n \right\rangle }{\norm{e_n}^2}$ -- коэффициент Фурье $x$ по системе $\{e_n\}$
	
	$\sum\limits_{n = 1}^\infty  c_n(x) e_n$ -- ряд Фурье $x$ по системе $\{e_n\}$
	
\end{definition}

\begin{remarks}\thmslashn
	
	\begin{enumerate}
		\item 
		Теорема утверждает, что если $x$ -- сумма ортогонального ряда, то это ряд Фурье
		
		
		\item 
		$x = c_ne_n + z$, где $z = \sum\limits_{k\not = n} c_ke_k$
		
		$c_ne_n$ -- проекция на $e_n$
		
	\end{enumerate}
	
\end{remarks}

\begin{theorem}[о частичных суммах ряда Фурье]\thmslashn 
	
	$x \in H \quad e_1, e_2, \ldots$ -- ортогональная система,  $S_n:= \sum\limits_{k = 1}^{n} c_k(x)e_k$

	Тогда
	
	\begin{enumerate}
		\item 
		$S_n$ -- проекция $x$ на $lin\{e_1, e_2, \ldots, e_n\} =:L_n$ \TODO какая-то очень красивая $L$
		
		\item
	 	$S_n$ -- наилучшее приближение к $x$ в $L-n$, т.ч. 
	 	
	 	$\norm{x-S_n} = \min\limits_{y \in L_n}\norm{x - y}$
	 	
	 	\item
	 	$\norm{S_n} \leqslant \norm{x}$
		
	\end{enumerate}
\end{theorem}

\begin{proof}\thmslashn

	\begin{enumerate}
		\item 
		$z:= x - S_n$ и покажем, что $z \prec L_n$
		
		Надо доказать, что $z\prec e_k$ при $k = 1, 2, \ldots, n$
		
		$\left\langle z, e_k \right\rangle = \left\langle x - S_n, e_k \right\rangle = \left\langle x, e_k \right\rangle - \left\langle \sum\limits_{j = 1}^{n}c_j(x)e_j, e_k \right\rangle = \left\langle x, e_k \right\rangle - \sum\limits_{j = 1}^{n} c_j(x)\left\langle e_j, e_k \right\rangle = \left\langle x, e_k \right\rangle - c_k(x)\left\langle e_k, e_k \right\rangle = 0$
		
		\item
		$x - S_n \prec y$, т.ч. $y \in L_n$
		
		$x - y = S_n + z - y = (S_n - y) + z$, при этом $S_n - y \in L_n$
		
		$\norm{x - y}^2 = \norm{S_n - y}^2 + \norm{z}^2 \geqslant \norm{z}^2 = \norm{x - S_n}^2$ и равенство, когда $y = S_n$
		
		\item
		$x = S_n + z$, где $S_n \prec z \Rightarrow \norm{x}^2 = \norm{S_n}^2 + \norm{z}^2 \geqslant \norm{S_n}^2$
		
	\end{enumerate}	

\end{proof}

\begin{consequence}[неравенство Бесселя]\thmslashn
	
	$\norm{S} \leqslant \norm{x}$
	
\end{consequence}

\begin{proof}\thmslashn
	
	$\norm{S_n} \leqslant \norm{x}$, но также $\norm{S_n} \to \norm{S}$
	
\end{proof}

\begin{theorem}[Рисса-Фишера]\thmslashn 
	
	$x \in H \quad e_1, e_2, \ldots$ -- ортогональная система
	
	\begin{enumerate}
		\item 
		Ряд Фурье для $x$ сходится
		
		\item
		$x = \sum\limits_{k = 1}^\infty c_n(x)e_n + z$, где $z \prec e_n \,\,\forall n$
		
		\item
		$x = \sum\limits_{k = 1}^\infty c_n(x)e_n \Leftrightarrow \norm{x}^2 = \sum\limits_{k = 1}^\infty c_n(x)^2e_n^2$ -- тождество Парсевая
		
	\end{enumerate}
\end{theorem}

\begin{proof}\thmslashn
	
	\begin{enumerate}
		\item 
		$\sum\limits_{n = 1}^\infty c_n(x)e_n$ -- ортогональный ряд сходится $\Leftrightarrow \sum\limits_{n = 1}^\infty \norm{c_n(x)e_n}^2 = \sum\limits_{n = 1}^\infty |c_n(x)|^2\norm{e_n}^2 < +\infty$
		
		$\sum\limits_{n = 1}^N |c_n(x)|^2 \norm{e_n}^2 = \norm{S_N}^2 \leqslant \norm{x}^2$
		
		\item
		$z := x - \sum\limits_{n = 1}^\infty c_n(x)e_n \quad \left\langle z, e_n \right\rangle = \left\langle x, e_n \right\rangle -  \sum\limits_{n = 1}^\infty \left\langle c_k(x)e_k, e_n \right\rangle = \left\langle x, e_n \right\rangle -  c_k(x) \left\langle e_n, e_n \right\rangle = 0$
		
		\item
		$z \prec \sum\limits_{n = 1}^\infty c_n(x)e_n \Rightarrow \norm{x}^2 = \norm{\sum c_n(x)e_n}^2 + \norm{z}^2 = \sum |c_n(x)|^2\norm{e_n}^2 + \norm{z}^2$ 
		
	\end{enumerate}	
	
\end{proof}

\begin{remarks}\thmslashn
	
	\begin{enumerate}
		\item 
		$\sum\limits_{n = 1}^\infty c_n(x)e_n$ -- ортогональная проекция на $\Cl Lin\{e_n\}$
		
		\item 
		Если $\sum c_n^2 < \infty$, то существует $x$, т.ч. $c_n(x) = c_n \forall n$
		
	\end{enumerate}
	
\end{remarks}

\begin{definition}\thmslashn
	
	Ортогональная система замкнута, если $\forall x\in H$ выполняется тождаство Парсеваля. 
	
	Ортогональная система полная, если не существует ненулевого $z$, т.ч. $z \prec e_n\,\,\forall n$
	
	Ортогональная система базис, если $\forall x \in H\quad x = \sum c_n(x) e_n$ 
	
\end{definition}

\begin{theorem}\thmslashn 
	
	$e_1, e_2, \ldots$ -- ортогональная система
	
	Тогда равносильны 
	
	\begin{enumerate}
		\item 
		Базис
		
		\item
		$\forall x, y \in H \quad \left\langle x, y\right\rangle = \sum\limits_{n = 1}^{\infty}c_n(x)\bar{c_n(y)} \norm{e_n}^2$
		
		\item
		Полная 
		
		\item
		Замкнутая
		
		\item
		$\Cl Lin\{e_n\} = H$
		
	\end{enumerate}
\end{theorem}

\begin{proof}\thmslashn
	
	\begin{enumerate}
		\item[2)$\Rightarrow$4)]
		Очев $x = y$
		
		\item[4)$\Rightarrow$1)]
		3 из теоремы
		
		\item[1)$\Rightarrow$2)]
		$x = \sum c_n(x)e_n$
		
		$y = \sum c_n(y)e_n$
		
		$\left\langle x, y\right\rangle = \sum\sum c_n(x)\bar{c_n(y)} \left\langle e_n, e_k\right\rangle $
		
		\item[1)$\Rightarrow$5)]
		$x = \sum c_n(x)e_n \in \Cl Lin\{e_n\}$
		
		\item[1)$\Rightarrow$3)]
		$z\prec e_n \Rightarrow z = \sum c_n(x)e_n = 0$
		
		\item[5)$\Rightarrow$3)]
		Если $z \prec e_n \,\,\forall n \Rightarrow z\prec Lin\{e_1, e_2, \ldots, e_n\} \Rightarrow z\prec \Cl Lin\{e_n\} = H \Rightarrow z\prec z \Rightarrow \left\langle z, z\right\rangle = 0 \Rightarrow z = 0$
		
		\item[3)$\Rightarrow$1)]
		$x = \sum c_n(x)e_n + z$, где $z \prec e_n \Rightarrow$ по 3 $z = 0 \Rightarrow e_n$ -- базис
		
		
	\end{enumerate}	
	
\end{proof}
