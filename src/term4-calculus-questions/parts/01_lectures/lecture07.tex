\Subsection{Лекция 7}

Суммирование рядов Фурье

\begin{definition}\thmslashn

	$\sum\limits_{n = 0}^{\infty} a_n \qquad A_n := \sum\limits_{k = 0}^{n} a_k \quad \alpha_n := \frac{A_0 + A_1 + \ldots + A_n}{n+1}$
	
	Если ряд сходится к $S$, то $A_n \to S \Rightarrow \alpha_n\to S$

	Это было суммирования ряда по Чезаро
	
\end{definition}

\begin{example}\thmslashn
	
	$1- 1 + 1 - 1+ \ldots$
	
	$A_n: \,\,1 , 0, 1, 0, 1, \ldots$
	
	$\alpha_{2n - 1} = \frac{1}{2}\,\,\,\alpha_{2n} - \frac{n + 1}{2n+1} \to \frac{1}{2}$
	
\end{example}

\begin{consequences}\thmslashn

	\begin{enumerate}
		\item 
		Линейность 
		
		$c\sum (\alpha a_n + \beta b_n) = c\alpha\sum a_n + c\beta \sum b_n$
		
		\item 
		Если ряд сходится, то он сходится по Чезаро к той же сумме
		
		\item
		Если ряд сходится по Чезаро, то $a_n = o(n)$
		
		\begin{proof}\thmslashn
			
			$\alpha_n \to S \Rightarrow \frac{n}{n+1} \alpha_{n-1} \to S \Rightarrow \alpha_n - \frac{n}{n+1} \alpha_{n-1} \to 0 \Rightarrow A_n = o(n)$
			
			$\Rightarrow a_n = A_n - A_{n-1} = o(n)$
			
		\end{proof}
	
		\item
		$\alpha_n = \sum\limits_{k = 0}^n \left( 1 - \frac{k}{n} \right)a_k$
		
		\begin{proof}\thmslashn
			
		$\alpha_n = \frac{A_0 + A_1 + \ldots + A_n}{n+1} = \frac{a_0 + (a_0 + a_1) + (a_0 + a_1 + a_2) + \ldots + (a_0 + a_1 + \ldots + a_n)}{n+1} = a_0 \frac{n+1}{n+1} + a_1\frac{n}{n+1} + \ldots + a_n\frac{1}{n+1}$
			
		\end{proof}
	
	\end{enumerate}

\end{consequences}

\begin{remark}\thmslashn

	Теорема Харди. Если $a_n = O(\frac{1}{n})$ и ряд сходится по Чезаро, то он сходится. (условие возрастания можно заменить на $a_n \geqslant - \frac{c}{n}$)

\end{remark}

\begin{example}\thmslashn
	
	$\frac{1}{2} + \sum\limits_{n = 1}^{\infty} \cos nt$
	
	Частичные суммы $D_n(t) := \frac{1}{2} + \sum\limits_{k = 1}^n \cos kt = \frac{\sin\left(n+\frac{1}{2}\right) t}{2 \sin \frac{t}{2}}$
	
	$\Phi_n(t) := \frac{D_0(t) + D_1(t) + \ldots + D_n(t)}{n+1}$ -- ядро Фейера
	
\end{example}

\begin{consequences}\thmslashn
	
	\begin{enumerate}
		\item 
		$\Phi_n(0) = \frac{\frac{1}{2} + \left(1 + \frac{1}{2}\right) + \ldots + \left(n + \frac{1}{2}\right)}{n+1} = \frac{1}{2} + \frac{n}{2} = \frac{n+1}{2}$
		
		\item
		При $t \not = 2nm \quad \Phi_n(t) = \frac{1}{n+1}\sum\limits_{k = 1}^{n}\frac{\sin(k + \frac{1}{2})t}{2 \sin \frac{t}{2}} = \frac{1}{(n+1)\cdot 2 \sin^2\frac{t}{2}} \cdot \frac{1}{2}(1 - \cos(n+1)t) = \frac{1}{2(n+1)}\cdot \frac{\sin^2 \frac{n+1}{n}t}{\sin^2 \frac{t}{2}}$
		
		\item
		$\Phi_n(t) \geqslant 0$
		
		\item
		$\max\limits_{\delta \leqslant |t| \leqslant \pi} \Phi_n(t) \leqslant \frac{1}{2(n+1) \sin^2\frac{\delta}{2}} \to 0$
		
		\item
		$\Phi_n$ -- четная функция
		
		\item
		$\int\limits_{-n}^n \Phi_n(t)\,dt = \pi$
		
	\end{enumerate}
	
\end{consequences}

Возьмем ряд Фурье для функции $f$

$S_n(x) = \frac{1}{\pi} \int\limits_{-n}^nD_n(t)f(x - t)\, dt$

$\sigma_n(x) = \frac{S_0(x) + S_1(x) + \ldots + S_n(x)}{n+1} = \frac{1}{\pi} \int\limits_{-n}^n\frac{D_0(x) + D_1(x) + \ldots + D_n(x)}{n+1} f(x - t) \,dt = \frac{1}{\pi}\int\limits_{-n}^n \Phi_n(t) f(x - t)\,dt$

\begin{definition}[Свертка функций]\thmslashn

	$f, g \in L^1[-\pi, \pi]$ продолжим до $2\pi$ - период
	
	$f*g(x) = \int\limits_{-n}^n f(t) g(x -t)\,dt$

\end{definition}

\begin{consequences}\thmslashn
	
	\begin{enumerate}
		\item 
		$f*g \in L^1[-\pi, \pi]$
		
		\begin{proof}\thmslashn
			
			$F(x, t) := f(t)g(x - t)$ -- измеримая функция
			
			$\int\limits_{[-\pi, \pi]^2}|F(x, t)|\,dx\,dt = \int\limits_{-\pi}^\pi |f(t)|\int\limits_{-\pi}^\pi |g(x - t)\,dx\,dt| = \int\limits_{-\pi}^\pi |f(t)|\,dt \cdot \int\limits_{-\pi}^\pi |g(x)|\,dx < +\infty$
			
			$\int\limits_{-\pi}^\pi |f * g(x)|\,dx = \int\limits_{-\pi}^\pi\abs{\int\limits_{-\pi}^\pi f(t)g(x - t)\,dt}\,dx \leqslant \int\limits_{-\pi}^\pi\int\limits_{-\pi}^\pi|F(x, t)|\,dt\,dx < +\infty$
			
		\end{proof}
	
		\item
		$f*g = g*f$
		
		\begin{proof}\thmslashn
			
			$\int\limits_{-\pi}^\pi f(t) g(x - t)\,dt = - \int\limits_{x + \pi}^{x- \pi} f(x - y)g(y)\,dt = \int\limits_{-\pi}^\pi f(x - y)g(y)\,dy = g*f(x)$
		
		\end{proof}
	
		\item
		$c_k(f*g) = 2\pi c_k(f)c_k(g)$
		
		\begin{proof}\thmslashn
			
			$c_k(f*g) = \frac{1}{\pi} \int\limits_{-\pi}^\pi  e^{-i k x} \int\limits_{-\pi}^\pi  f(t) g(x - t)\,dt \,dx = \frac{1}{2\pi} \int\limits_{-\pi}^\pi f(t)e^{-ikt}\left( \int\limits_{-\pi}^\pi g(x - t) e^{-ik(x - t)}\,dx \right)\,dt = \frac{2\pi c_k(g)}{2\pi} \int\limits_{-\pi}^\pi f(t)e^{-ikt}\,dt = 2\pi c_k(g)c_k(f)$
		
		\end{proof}
		
		\item
		Если $f\in L^p[-\pi, \pi]\,\,\frac{1}{p} + \frac{1}{q} = 1$, то $f*g \in C_{2\pi}$ и $\norm{f*g}_\infty \leqslant \norm{f}_p\norm{g}_q $
		
		\begin{proof}\thmslashn
			
			$\abs{f*g(x)} = \abs{\int\limits_{-\pi}^\pi  f(t) g(x - t) \, dt} \leqslant \int\limits_{-\pi}^\pi \abs{f(t)}\abs{g(x - t)}\,dt \leqslant \left(\int\limits_{-\pi}^\pi  |f(t)|^p\,dt \right)^{\frac{1}{p}} \left(\int\limits_{-\pi}^\pi  |g(t)|^q\,dt \right)^{\frac{1}{q}} = \norm{f}_p\norm{g}_q$
			
			$\abs{f*g(x + h) - f*g(x)} = \abs{\int\limits_{-\pi}^\pi  f(t) g(x + h - t) \, dt - \int\limits_{-\pi}^\pi  f(t) g(x - t) \, dt} \leqslant \int\limits_{-\pi}^\pi  \abs{f(t)}\cdot \abs{g(x + h - t) - g(x - t)}  \, dt \leqslant \norm{f}_p \left( \int\limits_{-\pi}^\pi |g(x + h - t) - g(x - t)|^q\,dt \right)^\frac{1}{q} = \norm{f}_p\cdot\norm{g_h - h}_q \tou{h\to 0} 0$
			
		\end{proof}
	
		\item
		$f \in L^p[-\pi, \pi], g \in L^1[-\pi, \pi]\quad p > 1$
		
		Тогда $\norm{f * g}_p \leqslant \norm{f}_p\norm{g}_1$
		
		\begin{proof}\thmslashn
			
			$\norm{f*g(x)}^p_p = \int\limits_{-\pi}^\pi  \abs{f*g(x)}^p \, dx = \int\limits_{-\pi}^\pi \abs{ \int\limits_{-\pi}^\pi f(t) g(x - t)\,dt}^p\,dx \leqslant \norm{g}_1^{\frac{p}{q}}  \int\limits_{-\pi}^\pi \int\limits_{-\pi}^\pi |f(t)|^p|g(x - t)|\,dt\,dx$
			
			$\abs{ \int\limits_{-\pi}^\pi f(t) g(x - t)\,dt} \leqslant  \int\limits_{-\pi}^\pi |f(t)||g(x - t)|^{\frac{1}{p}}\cdot |g(x - t)|^{\frac{1}{q}}\,dt \leqslant \left(\int\limits_{-\pi}^\pi |f(t)|^p |g(x - t)|\,dt \right)^{\frac{1}{p}}\left(\int\limits_{-\pi}^\pi |g(x - t)|\,dt \right)^{\frac{1}{q}} = $
			
			$q$ из Гёльдера такая, что $\frac{1}{p} + \frac{1}{q} = 1$
			
			 $\norm{g}^{frac{p}{q}}_1 \cdot\norm{f}^p_p\cdot\norm{g}_1 = \norm{f}^p_p\norm{g}^p_1$
			
		\end{proof}
		
	\end{enumerate}
	
\end{consequences}

\begin{definition}[Аппроксимативная единица $K_n(t)$]\thmslashn

	$h \in D$ -- множество параметров $\quad h_0$ -- предельная точка $D$
	
	\begin{enumerate}
		
		\item
		$K_n \in K^{1}[-\pi, \pi]$
		
		\item
		$\int\limits_{-\pi}^\pi K_n(t)\,dt = 1$
		
		\item
		$\norm{K_n}_1 = \int\limits_{-\pi}^\pi |K_n(t)|\,dt \leqslant M$ (не зависит от $h$)
		
		\item
		$\int\limits_{[-\pi, \pi]\setminus (-\delta, \delta)} \abs{K_n(t)}\,dt \tou{h \to h_0} 0$
		
		\item[4']
		$\ess\sup\limits_{[-\pi, \pi]\setminus (-\delta, \delta)} \abs{K_n(t)} \tou{h \to h_0} 0$
	
	\end{enumerate}

	$1 + 2 + 3 + 4$ -- аппроксимативная единица
	
	$1 + 2 + 3 + 4'$ -- усиленная аппроксимативная единица

\end{definition}

\begin{example}\thmslashn

	$\frac{1}{\pi}\Phi_n$ -- усиленная аппроксимативная единица
	
\end{example}

\begin{theorem}[об аппросмимативной единице]\thmslashn

	$K_n$ -- аппроксимативная единица
	
	\begin{enumerate}
		\item 
		Если $f\in C_{2\pi}$, то $f*K_n \toto{h \to h_0} f $
		
		\item
		Если $f \in L^p[-\pi, \pi]$, $1 \leqslant p < +\infty$, то $\norm{f* K_n - f}_p \tou{h \to 0} 0$
		
		\item
		Если $f \in L^1[-\pi, \pi]$ и непрерывная в точке $x$
		
		$K_n$ -- усиленная аппроксимативная единица, то $f*K_n(x) \tou{h \to h_0} f(x)$
		
	\end{enumerate}
\end{theorem}

\begin{proof}\thmslashn
	
	$f*K_n(x) - f(x) = \int\limits_{-\pi}^\pi K_n(t) f(x - t)\,dt - \int\limits_{-\pi}^\pi K_n(t) f(x)\,dt = \int\limits_{-\pi}^\pi K_n(t) (f(x - t) - f(x))\,dt $

	\begin{enumerate}
		\item 
		Берем по $\varepsilon$ из равномерной непрерывности $\delta > 0$
		
		Если $|t| < \delta$, то $|f(x - t) - f(x)| < \varepsilon\qquad |f(t)| \leqslant A$
		
		$\int\limits_{-\pi}^\pi \abs{K_n(t)\left(f(x - t) - f(t)\right)}\,dt = \int\limits_{-\delta}^\delta + \int\limits_{[-\pi, \pi]\setminus (-\delta, \delta)}$
		
		$\int\limits_{-\delta}^\delta < \varepsilon \int\limits_{-\delta}^\delta |K_n(t)|\,dt \leqslant \varepsilon \int\limits_{-\pi}^\pi |K_n(t)|\,dt \leqslant M\varepsilon$
		
		$\int\limits_{[-\pi, \pi]\setminus (-\delta, \delta)} \leqslant 2A \int\limits_{[-\pi, \pi]\setminus (-\delta, \delta)} |K_n(t)|\,dt \to 0$
		
		\item
		Берем $\delta$ в точке $x$
		
		$\int\limits_{[-\pi, \pi]\setminus (-\delta, \delta)} \leqslant \ess\sup\limits_{t\in [-\pi, \pi]\setminus (-\delta, \delta)}|K_n(t)|\cdot \int\limits_{[-\pi, \pi]\setminus (-\delta, \delta)} |f(x - t) - f(x)|\,dt \leqslant \int\limits_{-\pi}^\pi \left(|f(x - t)| + |f(t)|\right)\,dt = 2\int\limits_{-\pi}^\pi |f(t)|\,dt$
	
	\end{enumerate}
	
	
\end{proof}



\begin{definition}\thmslashn
\end{definition}

\begin{lemma}\thmslashn
\end{lemma}

\begin{theorem}\thmslashn
\end{theorem}

\begin{proof}\thmslashn
\end{proof}

\begin{consequence}\thmslashn
\end{consequence}
